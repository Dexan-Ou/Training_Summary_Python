% XeLaTeX

\documentclass{article}
\usepackage{ctex}
\usepackage{xypic}
\usepackage{amsfonts,amssymb}
\usepackage{multirow}
\usepackage{geometry}
\usepackage{graphicx}
\usepackage{listings}
\usepackage{lipsum}
\usepackage{courier}
\usepackage{fancyvrb}
\usepackage{etoolbox}


\linespread{1.2}
\geometry{left=3cm,right=2.5cm,top=2.5cm,bottom=2.5cm}

\makeatletter
\patchcmd{\FV@SetupFont}
  {\FV@BaseLineStretch}
  {\fontencoding{T1}\FV@BaseLineStretch}
  {}{}
\makeatother

\lstset{basicstyle=\small\fontencoding{T1}\ttfamily,breaklines=true}
\lstset{numbers=left,frame=shadowbox,tabsize=4}
%\lstset{extendedchars=false}
\begin{document}

\title{Python 课程项目:ACM 训练总结系统}
\author{王凯祺、刘梓晖、田启睿、区炜彬}
\maketitle

\section{开发背景}

中山大学 ACM 集训队目前正在使用原有的 ACM 训练总结系统( http://sysuteam.com )。这个系统能帮助教练了解集训队的做题情况。然而,这个系统有一些不足之处,例如每支队伍没有自己的主页,队伍不能在上面写总结、发题解。

我参考了一下浙大的队伍主页,包含队伍信息、训练帐号、目标、友情链接、杂事杂项(一些小结和提醒)、板子整理、个人训练、团队训练、比赛记录等信息。他们的队伍主页支持 MarkDown 语法,并且有版本控制(即可以恢复到历史版本)。对于每场比赛,他们都有总结。我们决心做出跟浙大一样好的 ACM 训练总结系统。

\section{项目地址}

本项目开源,项目地址为: https://github.com/africamonkey/Training\_Summary\_Python

\section{项目成员}

王凯祺 16337233

刘梓晖 16337166

田启睿 16337222

区炜彬 16337202

\section{拟定功能}

比赛的创建与管理

队伍管理

队伍主页(含版本控制),底部包含该队伍参与的所有比赛及总结

比赛列表,及该比赛下每支队的做题情况

\section{实现方法}

主要使用语言: Python3

Web 应用框架:Django

数据库:Sqlite3

\end{document}
















